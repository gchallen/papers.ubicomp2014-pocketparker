\section{Introduction}

Drivers enter a parking lot without having any specific knowledge about the
lot's availability. Especially in urban areas, it is common for any driver to
spend an additional, non-trivial amount of time looking for a parking spot. This
searching time not only translates into frustration but also wastes energy and
produces carbon dioxide harmful for the environment.

In order to assist drivers searching for a parking space, numerous smartphone
applications are developed and available in the online application stores such
as Google Play Store and App Store. Although many drivers find these smartphone
applications useful, these applications either do not provide real-time parking
lot availability or just displays publicly-accessible availability information.

To overcome this limitation, a few academic systems have been
proposed~\cite{4212497, Chen:2012:COS, Delot:2009:CRP, 5062057,
Mathur:2010:PDS}. However, these systems make a variety of assumptions that
prevent them from being immediately deployable. Typical assumptions include the
availability of additional infrastructure~\cite{5062057}, additional equipment
deployed on vehicles~\cite{Mathur:2010:PDS}, the presence of a vehicular
network~\cite{Delot:2009:CRP, Mathur:2010:PDS, 4212497}, and manual inputs from
users~\cite{Chen:2012:COS}.

We present {\it PocketParker}, a system that predicts parking lot availability
using smartphones. Unlike previous approaches, our goal is to not require any
additional input or infrastructure other than the smartphones used by the users
of PocketParker. More specifically, we only require our users to download a
smartphone application that runs in the background and detects parking related
events. Using this information, we construct a prediction model that estimates
the availability of a parking lot in our backend server. This simple operational
requirement gives PocketParker the advantage of being easily deployable. In
general, we consider our approach to be an example of a special type of
crowdsourcing that does not require any manual user input. We term {\it
pocketsourcing} to refer to this type of crowdsourcing.

Our goal of pocketsourcing accompanies two technical challenges. The first
challenge is detecting parking related events accurately while minimizing energy
consumption; the second challenge is accurately predicting parking lot
availability in the presence of {\it hidden drivers} who do not use our
application. We address the first challenge by designing a simple, yet effective
event detector; our event detector mainly relies on the accelerometer on a
smartphone to detect park and depark events, which is more energy-efficient than
GPS. We address the second challenge by designing an availability estimator that
probabilistically predicts the availability of a parking lot based on the events
reported by our event detector; our prediction model estimates arrival and
departure rates for a parking lot and adjusts the availability probability
accordingly.

To validate the effectiveness of PocketParker, we evaluate each of our
techniques in a setting best suited for the technique. We evaluate our parking
event detector in a controlled environment with eight volunteers participating
in ten parking scenarios. We design a simulator to evaluate our parking
availability estimator, which gives us the flexibility to experiment with a
variety of parameters and parking lot types. Finally, we evaluate the overall
effectiveness of PocketParker by deploying it on five smartphones used by our
participants over ten days. To compare our data to ground truth, we deploy four
cameras that monitor four parking lots over two weeks. We hand-code one day's
worth of images for two parking lots to measure the true availability.

\begin{figure}
\centering
\includegraphics[width=\columnwidth]{./figures/blockdiagram.pdf}

\caption{\textbf{The PocketParker architecture.} Events generated by an
activity detector running quietly on each smartphone are processed by a
central server and used to estimate parking lot availability.}

\label{fig-arch}
\end{figure}
