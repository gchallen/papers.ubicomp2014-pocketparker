\section{Introduction}

Drivers enter a parking lot without having any specific knowledge about the
lot's availability. It is not uncommon to spend an additional, non-trivial
amount of time looking for a parking spot. This searching time directly
translates into wasted gas and carbon dioxide harmful for the environment.

In order to assist drivers searching for a parking space, numerous systems and
smartphone applications have been proposed~\cite{Mathur:2010:PDS, 5062057,
Delot:2009:CRP, Chen:2012:COS, 4212497, open-spot}. 

We present {\it PocketParker}, a system that predicts parking lot availability
using smartphones. Unlike previous approaches, our goal is to not require any
additional input or infrastructure other than the smartphones used by the users
of PocketParker. In order to accomplish this goal, we address two main
challenges. We accomplish this goal by combining an activity detector
deployed as a smartphone appplication that automatically extracts park and
depark events as well as a prediction model that estimates parking lot
availability based on observed events. Our activity detector addresses the
challenges of energy efficiency and accuracy of detection. Our prediction model
addresses the challenges of estimating the number of {\it hidden drivers} who do
not participate in PocketParker as well as arrival and departure rates of a
parking lot. Using this information, PocketParker probabilistically estimates
how many parking spots are available at any given time.

\begin{figure}
\centering
\includegraphics[width=\columnwidth]{./figures/blockdiagram.pdf}

\caption{\textbf{PocketParker architecture.}}

\label{fig-arch}
\end{figure}

\subsection{Usage Model}
