\section{Introduction}

Parking lots present a difficult search problem. Drivers lack the visibility
to determine where spots are available, and may spend a non-trivial amount of
time searching for a spot. The problem is difficult enough that WikiHow
includes directions~\cite{wikihow-park}, and the Wall Street Journal has
published an online article~\cite{wsj-park} with tips on spot stalking for
shoppers during the holidays. Searching not only generates frustration but
also wastes energy and produces harmful carbon emissions.

Online smartphone application stores such as Google Play and the App Store
are teeming with apps claiming to help you find a parking spot. Although some
drivers may find these applications useful, they either do not provide
real-time parking lot availability or simply display publicly-available
information. Several research projects have attempted to address these
limitations~\cite{4212497, Chen:2012:COS, Delot:2009:CRP, 5062057,
Mathur:2010:PDS}, but include requirements rendering them impractical, such
as additional infrastructure~\cite{5062057}, on-vehicle
equipment~\cite{Mathur:2010:PDS} or vehicular
networking~\cite{Delot:2009:CRP, Mathur:2010:PDS}, or onerous manual user
input~\cite{Chen:2012:COS}. In contrast to these efforts, we believe the
solution to finding a parking spot is already in your pocket.

We present \textit{PocketParker}, a system that predicts parking lot
availability using smartphones and pocketsourcing. Unlike previous
approaches, PocketParker requires no additional infrastructure, no vehicle
modifications, and even no user input, only some small percentage of the
100~million smartphones already in use in the US~\cite{smartphone-numbers}.

specifically, we only require our users to download a smartphone application
that runs in the background and detects parking related events. Using this
information, we construct a prediction model that estimates the availability
of a parking lot in our backend server. This simple operational requirement
gives PocketParker the advantage of being easily deployable. In general, we
consider our approach to be an example of a special type of crowdsourcing
that does not require any manual user input. We term {\it pocketsourcing} to
refer to this type of crowdsourcing.

Our goal of pocketsourcing accompanies two technical challenges. The first
challenge is detecting parking related events accurately while minimizing energy
consumption; the second challenge is accurately predicting parking lot
availability in the presence of {\it hidden drivers} who do not use our
smartphone application. We address the first challenge by designing a simple,
yet effective event detector; our event detector mainly relies on the
accelerometer on a smartphone to detect park and depark events, which is more
energy-efficient than GPS. We address the second challenge by designing an
availability estimator that probabilistically predicts the availability of a
parking lot based on the events reported by our event detector; our prediction
model estimates arrival and departure rates for a parking lot and adjusts the
availability probability accordingly.

To validate the effectiveness of PocketParker, we evaluate each of our
techniques in a setting best suited for assessing the technique. We evaluate
our parking event detector in a controlled environment with eight volunteers
participating in ten parking scenarios. We design a simulator to evaluate our
parking availability estimator, which gives us the flexibility to experiment
with a variety of parameters and parking lot types. Finally, we evaluate the
overall effectiveness of PocketParker by deploying it with five smartphones used
by our participants over ten days. To obtain ground truth, we deploy four
cameras that monitor four parking lots over two weeks. We inspect and hand-code
one day's worth of images for two parking lots to measure their true
availability. Altogether, our results show the efficiency and accuracy of
PocketParker.

The rest of our paper is organized as follows. We start by presenting related
work in Section~\ref{sec-related} in order to distinguish PocketParker from
multiple previous efforts at parking monitoring. Next, in
Sections~\ref{sec-detector}~and~\ref{sec-model} we describe the two major
components of PocketParker: our parking event detector and availability
model. Our evalution in Section~\ref{sec-evaluation} uses several approaches
to evaluate PocketParker, including controlled experiments, simulations, and
a small deployment. Finally, we discuss limitations and future work in
Section~\ref{sec-future} before concluding in Section~\ref{sec-conclusions}.

\begin{figure}
\centering
\includegraphics[width=\columnwidth]{./figures/blockdiagram.pdf}

\caption{\textbf{The PocketParker architecture.} Events generated by an
activity detector running quietly on each smartphone are processed by a
central server and used to estimate parking lot availability.}

\label{fig-arch}
\end{figure}
