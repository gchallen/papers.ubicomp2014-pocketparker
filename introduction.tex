\section{Introduction}


Parking lots present a difficult search problem. Drivers lack the visibility
to determine where spots are available, and may spend a non-trivial amount of
time searching for a spot. The problem is difficult enough that WikiHow
includes directions~\cite{wikihow-park} and the Wall Street Journal has
published an online article~\cite{wsj-park} with tips on spot stalking for
shoppers during the holidays. Searching not only generates frustration but
also wastes energy and produces harmful environmental emissions.

-instrumented approach (embedded sensors, programmable signs, personnel):  not
    at all novel -- used for years in high demand, high churn areas (airports).
    OK.  But this is typically not used in situations where parking saturation
    occurs only intermittently, if predictably (universities during terms during
    peak class hours, suburban malls during holiday shopping seasons).  In these
    cases, the bulk of the time parking goes fallow.  Infrastructure costs are
    not justified.

-installation and maintenance:  hassle and expense

-keep a running total:  nice theory...
   -cost in academic study -- cheap, but with cheap infrastructure.  Cite
    Real World costs of ticket dispensers (1k) or booths (5k) for permanant structures
    and communication (5k / line hardwired -- M/STP) to get info back to server
    for broadcast
   -campus lots typically have several entrancess -- i.e., need several sensors
   -still need a means to communicate findings back to end users:  either
    programmable signs (50k each -- M/STP) or some form of mobile sourcing (i.e., ours)
   -if it is so simple, why don't more lots take this approach?

Alternatively, need some other way of keeping track of whether a lot (or spot)
    is free.  Specifically, payment forms.

-related approaches to ensuring parking:  ensuring spot availability via payment
   -Pay stations:  ca. 90k / station in A**2
   -Pay booths:  ca. 5k / booth (cite)

***ADD IN CITATIONS...

Online smartphone application stores such as Google Play and the App Store
teem with apps to locate parking spots. Although some drivers may find these
applications useful, the apps either do not provide real-time parking lot
availability or simply display publicly-available information. Several
research projects have attempted to address these limitations~\cite{4212497,
Chen:2012:COS, Delot:2009:CRP, 5062057, Mathur:2010:PDS} but include
requirements rendering them impractical.  They either require additional
infrastructure~\cite{5062057}, on-vehicle equipment~\cite{Mathur:2010:PDS}
or networking~\cite{Delot:2009:CRP, Mathur:2010:PDS}, or onerous manual user
input~\cite{Chen:2012:COS}. In contrast, we believe the solution is already
in peoples' pockets.

We present \textit{PocketParker}, a system that predicts parking lot
availability using smartphones. Unlike previous approaches, PocketParker
requires no additional infrastructure, no vehicle modifications, and no user
input, only installation on a smartphone.  PocketParker runs unattended in
the background and uses the accelerometer to detect parking lot
arrivals and departures.  These are forwarded to a central server, which
incorporates them into per-lot availability models.  This allows PocketParker
to order lots accurately by the probability that they contain an available
spot.  In general, we consider our approach to be an example of a subset of
crowdsourcing that does not require any manual user input, which we call
\textit{pocketsourcing}.

Predicting parking availability requires the efficient and accurate detection
of parking-related events and the incorporation of the effect of
\textit{hidden drivers}---drivers not using PocketParker---into our
availability model. We address the first challenge with a simple yet effective
and energy conserving event detector which uses sense data to record vehicle
arrivals and departures.  The second goal we achieve with an availability
estimator that maintains a probability model for each lot by incorporating
data from PocketParker clients. We use detected events both to estimate
arrival and departure rates.  Even with limited information, there are moments
when PocketParker can be certain about the availability of a parking spot in a
given lot and can use this certainty to assist users find spots.

We perform an evaluation of PocketParker using a variety of methods
tailored to each system component. We evaluate our parking event detector in
a controlled environment with eight volunteers participating in ten parking
scenarios. We design a simulator to evaluate our parking availability
estimator, giving us the flexibility to experiment with a variety of
parameters and parking lot types. We evaluate the overall effectiveness of
PocketParker in a field test involving 105 smartphones users over forty five
days. To obtain ground truth, we deploy four cameras that monitor two parking
lots over two weeks. We inspect and hand-code four days' worth of images of
these lots to measure their true availability. Our results demonstrate the
overall efficiency and accuracy of PocketParker.

PocketParker has several components distributed across users' smartphones and
a backend server. Our paper describes each component in detail. We start by
presenting related work in order to distinguish PocketParker from previous
parking monitoring approaches. The next two sections describe two major
components of PocketParker: our parking event detector and availability model.
The evaluation that follows is based both on simulations, controlled
experiments, and a prototype deployment involving 105 users.  Finally, we
discuss limitations and future work.

\begin{comment}
\begin{figure}
\centering
\includegraphics[height=1.5in]{./figures/blockdiagram.pdf}

\caption{\textbf{The PocketParker architecture.} Events generated by an
activity detector running quietly on each smartphone are processed by a
central server and used to estimate parking lot availability.}

\label{fig-arch}
\vspace*{-0.2in}
\end{figure}
\end{comment}
