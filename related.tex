\section{Related Work}
\label{sec-related}

\subsection{Activity Detection}

\subsection{Parking Lot Monitoring}

There are a large number of parking lot applications available in the online
smartphone application stores. A typical parking lot application provides
location as well as pricing information and allow reservation of a parking spot.
Some applications also report parking lot availability based on publicly
available information. To the best of our knowledge, there is no application
that automatically infers parking lot availability by monitoring drivers.

Most close to our work is ParkNet~\cite{Mathur:2010:PDS}, a system that
estimates street parking availability. ParkNet uses vehicles equipped with GPS
and a ultrasonic range finder that scan the surrounding areas and detect
empty street parking spots. Compared to ParkNet, our approach relies only on
smartphones and do not require any additional input.

A few systems have been proposed to assist parking using vehicular ad-hoc
networks or crowdsourcing. Delot et al.~\cite{Delot:2009:CRP} propose a parking
lot reservation system in a vehicular network. Chen et al.~\cite{Chen:2012:COS}
proposes a crowdsourcing approach that asks participants to report the
surrounding parking availability. Caliskan et al.~\cite{4212497} propose an
availability prediction model based on information exchanged by vehicles in a
vehicular ad-hoc network. Their approach assumes that for each parking lot,
vehicles that drive by the parking lot receive the parking lot information such
as the capacity, the occupancy, the arrival rate, and the departure rate. Since
the propagation delay in a vehicular network makes this information stale, a
prediction model is used to estimate current availability. In contrast to
our work, this approach assumes that the necessary information is initially
accurately measured at the parking lot.

\subsection{Transportation-Related Projects}

A few previous systems have been proposed to enable automatic transit
tracking~\cite{Biagioni:2011:EAT, Thiagarajan:2010:CTT, Zhou:2012:LWP}. From the
high-level point of view, some of the techniques such as activity detection and
location tracking that transit tracking requires bear similarities to
our techniques; however, these techniques need to be optimized and tailored
towards different scenarios, hence the specifics vary widely.

Thiagarajan et al.~\cite{Thiagarajan:2010:CTT} propose an automatic, real-time
transit tracking approach that uses smartphones of public transit riders as data
sources. They propose an algorithm to detect when a user is traveling in a
vehicle and an algorithm to detect if a vehicle is a public transit vehicle.
EasyTracker~\cite{Biagioni:2011:EAT} uses smartphones deployed in buses to
enable automatic transit tracking. The goal of EasyTracker is to require no
other input than what is from the deployed smartphones. The system combines a
few mechanisms to realize this goal such as route extraction, stop extraction,
and arrival time prediction. Zhou et al.~\cite{Zhou:2012:LWP} propose a bus
arrival time estimation system based on smartphones used by public transit
riders. They combine multiple sources such as accelerometer data, audio, and
cell tower signals to detect if a rider is in a public transit vehicle and if
so, which bus it is.

VTrack~\cite{Thiagarajan:2009:VAE}

StarTrack~\cite{Ananthanarayanan:2009:SFE}
