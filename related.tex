\section{Related Work}
\label{sec-related}

\subsection{Activity Detection}

\subsection{Parking Lot Monitoring}

There are a large number of parking lot applications available in the online
smartphone application stores. A typical parking lot application provides
location as well as pricing information and allow reservation of a parking spot.
Some applications also report parking lot availability based on publicly
available information. To the best of our knowledge, there is no application
that automatically infers parking lot availability by monitoring drivers.

Most close to our work is ParkNet~\cite{Mathur:2010:PDS}, a system that
estimates street parking availability. ParkNet uses vehicles equipped with GPS
and a ultrasonic range finder that scan the surrounding areas and detect
empty street parking spots. Compared to ParkNet, our approach relies only on
smartphones and do not require any additional input.

A few systems have been proposed to assist parking using vehicular ad-hoc
networks or crowdsourcing. Delot et. al~\cite{Delot:2009:CRP} propose a parking
lot reservation system in a vehicular network. Chen et. al~\cite{Chen:2012:COS}
proposes a crowdsourcing approach that asks participants to report the
surrounding parking availability. Caliskan et. al~\cite{4212497} propose an
availability prediction model based on information exchanged by vehicles in a
vehicular ad-hoc network. This model is close to the model we use in this
paper; for each parking lot, vehicles that drive by the parking lot receive the
parking lot information such as the capacity, the occupancy, the arrival rate,
and the departure rate. Due to the propagation delay in a vehicular network,
this information can be stale. Hence, the model is used to estimate the current
availability. In contrast to our approach, the model assumes that the necessary
information is initially accurately measured at the parking lot.

\subsection{Transportation-Related Projects}
