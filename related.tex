\section{Motivation and Related Work}

While infrastructure solutions for monitoring lot availability exist, they
are extremely expensive. The SFPark system spent \$18 million to instrument
\num{7000} street spots, or roughly \$\num{2500} per spot~\cite{sfpark}.
Surface lots are cheaper to monitor since equipment can be deployed only at
ingress-egress points, but the technology required to do so remains
expensive. The vehicle detector and transponder required at each entrance
costs \$\num{9700}~\cite{car-detect} and programmable sign to communicate lot
availability to drivers running \$\num{49000}~\cite{mstp-park}, not including
the continuing cost of telemetry. Our campus with 40~lots and over 80~lot
entrances would cost \$\num{776000} for entrance monitors alone, and over
\$2~million dollars with lot availability signs. Even using a
\$\num{350}-per-entrance research prototype based on wireless sensor network
nodes would cost \$\num{28000}, again not including the cost of
communication. These prohibitive costs are why, despite being commercially
available for years, most parking lots are still unmonitored and drivers left
to search them unaided.

As smartphones have become ubiquitous, multiple apps and research projects
have attempted to harness their capabilities to aid the parking process. But
while app marketplaces such as the Google Play Store teem with
parking-related apps, these apps either do not provide real-time parking lot
availability or simply display publicly-available information. Several
research projects have attempted to address these limitations but suffer from
limitations that prevent them from scaling, requiring additional
infrastructure~\cite{5062057}, on-vehicle equipment~\cite{Mathur:2010:PDS}
vehicle-to-vehicle networking~\cite{Delot:2009:CRP, Mathur:2010:PDS}, or
onerous manual user input~\cite{Chen:2012:COS}. To the best of our knowledge,
PocketParker is the first app that can monitor parking lot availability
without interacting with users.

Most close to our work is Parksense~\cite{Nawaz:2013:PSB}, a system that
leverages the ubiquity of Wifi beacons to monitor on-street parking
availability. Our study, borne out of suburban campus locale, must cope with
alternative sensing mechanisms in the wake of no proximate Wifi signals. We
have also tuned our tracking and reporting methodology to address the
different challenges produced by lot, rather than street, parking.
Parksense's reliance on Wifi signals can only produce significant delay in
detecting departure events, making the data less useful to other drivers.
ParkNet~\cite{Mathur:2010:PDS} is another system that estimates street
parking availability. ParkNet uses vehicles equipped with GPS and as
ultrasonic range finder that scans surrounding areas to detect empty street
parking spots. Compared to ParkNet, our approach does not require additional
hardware or vehicle capabilities.

PocketParker's parking detector builds on existing approaches to accurate
and energy-efficient activity recognition~\cite{Constandache:2010:DYS,
Keally:2011:PTP, Reddy:2010:UMP, Yang:2011:DDP, Wang:2009:FEE}. While our
current detector is both simple and parking-focused, continued progress in
reducing the energy overhead and increasing the accuracy of smartphone
activity recognition algorithms will improve PocketParker's performance.
