\section{Related Work}
\label{sec-related}

\subsection{Activity Detection}

Activity detection is a basic primitive that enables other higher-level
functionalities. Several previous systems~\cite{Constandache:2010:DYS,
Keally:2011:PTP, Reddy:2010:UMP, Yang:2011:DDP, Wang:2009:FEE} have proposed
techniques for activity detection, solving the main challenges of energy
efficiency and accuracy. Yang et al.~\cite{Yang:2011:DDP} develop an algorithm
that detects if a driver is using a phone by sending high-frequency beeps via
in-car speakers. EEMSS~\cite{Wang:2009:FEE} is a system that provides continuous
identification of general human states such as walking, driving, and being in an
office. Reddy et al.~\cite{Reddy:2010:UMP} propose a classification approach
that determines human movement states such as walking, running, biking, or
vehicle-traveling. Constandache et al.~\cite{Constandache:2010:DYS} use 
smartphone accelerometers to determine users' walking trails. Keally et
al.~\cite{Keally:2011:PTP} use a combination of on-body wireless sensors and
smartphones to classify human states. Our system can benefit from these
techniques for detecting parking-related events; however, we find it sufficient
to use a simple detector since it avoids the complexity of detecting events
unrelated to parking.

\subsection{Parking Lot Monitoring}

There are a large number of parking lot applications available in the online
smartphone application stores. A typical parking lot application provides
location as well as pricing information and allow reservation of a parking spot.
Some applications also report parking lot availability based on publicly
available information. To the best of our knowledge, there is no application
that automatically infers parking lot availability by monitoring drivers.

Perhaps closest to our work is ParkNet~\cite{Mathur:2010:PDS}, a system that
estimates street parking availability. ParkNet uses vehicles equipped with GPS
and an ultrasonic range finder that scan the surrounding areas and detect
empty street parking spots. Compared to ParkNet, our approach relies only on
smartphones and does not require any additional input.

A few systems have been proposed to assist parking via vehicular ad-hoc networks
or crowdsourcing. SPARK~\cite{5062057} employs extra roadside infrastructure
that monitors parking lots and communicates with vehicles. Delot et
al.~\cite{Delot:2009:CRP} propose a parking lot reservation system in a
vehicular network. Chen et al.~\cite{Chen:2012:COS} propose a crowdsourcing
approach that asks participants to report the parking availability of their
surrounding areas. Caliskan et al.~\cite{4212497} propose an availability
prediction model based on information exchanged by vehicles in a vehicular
ad-hoc network. Their approach assumes that, for each parking lot, vehicles that
drive by the parking lot receive the parking lot information such as capacity,
occupancy, arrival rate, and departure rate. Since the propagation delay in a
vehicular network makes this information stale, a prediction model is used to
estimate current availability. In contrast to our work, their approach assumes
that each parking lot accurately measures the necessary information.

Google has operated a service called Open Spot for roughly two years but has
terminated it in 2012~\cite{open-spot}. It was a crowdsourcing approach that
relied on reports from users on free parking spots. The exact reason for this
discontinuation of service is not reported.

\subsection{Tracking-Related Projects}

A few previous systems have investigated techniques for automatic transit
tracking~\cite{Biagioni:2011:EAT, Thiagarajan:2010:CTT, Zhou:2012:LWP}. From a
high-level point of view, some of the techniques such as activity detection and
location tracking that transit tracking requires bear similarities to
our techniques; however, these techniques need to be optimized and tailored
towards different scenarios, hence the specifics vary widely.

Thiagarajan et al.~\cite{Thiagarajan:2010:CTT} propose an automatic, real-time
transit tracking approach that uses smartphones of public transit riders as data
sources. They propose an algorithm to detect when a user is traveling in a
vehicle and an algorithm to detect if a vehicle is a public transit vehicle.
EasyTracker~\cite{Biagioni:2011:EAT} uses smartphones deployed in buses to
enable automatic transit tracking. The goal of EasyTracker is to require no
other input than what is available from the deployed smartphones. The system
combines a few mechanisms to realize this goal such as route extraction, stop
extraction, and arrival time prediction. Zhou et al.~\cite{Zhou:2012:LWP}
propose a bus arrival time estimation system based on smartphones used by public
transit riders. They combine multiple sources such as accelerometer data, audio,
and cell tower signals to detect if a rider is in a public transit vehicle and
if so, which bus it is.

Other systems have used smartphone sensors to enable a variety of tracking
tasks. VTrack~\cite{Thiagarajan:2009:VAE} is a traffic monitoring system that
combines readings from multiple sensors for travel time estimation.
Nericell~\cite{Mohan:2008:NRM} uses multiple sensors to monitor road and traffic
conditions in an urban setting. P$^{2}$~\cite{Eriksson:2008:PPU} uses vibration
and GPS sensors to monitor road surface conditions. Balan et
al.~\cite{Balan:2011:RTI} use GPS devices deployed in taxi cabs to estimate
taxi fare and trip duration. StarTrack~\cite{Ananthanarayanan:2009:SFE,
Haridasan:2010:SNG} provides general abstractions for appliations that need
tracking functionalities such as recording, comparing, and querying tracks.
SignalGuru~\cite{Koukoumidis:2011:SLM} uses smartphones with cameras mounted in
vehicles to learn traffic signal schedule patterns, so drivers can adjust their
speed.
