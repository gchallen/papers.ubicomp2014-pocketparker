\section{Limitations and Future Work}

The pocketsourcing approach taken by PocketParker makes it easy to integrate
into existing mapping applications, which would provide access to the estimated
half-billion smartphone users that have installed Google Maps. The increase in
the monitored fraction this would represent would significantly improve
PocketParker's accuracy and usability. Integrating PocketParker with a mapping
application would permit PocketParker to access to frequent GPS updates it
triggers, allowing PocketParker to provide information about exact spot
location and not only per-lot availability.

PocketParker presently bases its parking predictions on a fifteen-minute
limited rolling window of recent parking events. We do not presently tap the
benefit of daily and weekly patterns that would otherwise enhance predictive
accuracy, but hope to do so in the future. Maintaining a database of
previously collected historical data from our own application would increase
the sample size and hence statistical accuracy of our parking predictions.
This is another area where integration with a mapping application would help,
providing PocketParker with access to much more data.

Presently, PocketParker supports a single user per vehicle. Proximity detection
of users using sensing would potentially allow the system to detect the case of
multiple users in a vehicle and thus to reduce spurious arrival and departure
events.

Finally, we believe that once users begin interacting with PocketParker we
will see different parking preferences emerge. Some user will want
PocketParker to help them aggressively hunt for spots, and be willing to wait
for drivers to leave. Others may be more interested in simply finding a spot
quickly even if it is farther away. PocketParker has several parameters that
can control its predictions, and we will need to determine how to expose
these options to users.

%
%An anticipated future enhancement of our application, adding capacity to
%solicit the exact preferred parking destination from a user, will also
%significantly benefit from historical data. We will be able to suggest lots
%available to end users based on proximity to logistically preferable but full
%lots. If a user further furnishes insight into how long he is willing to wait
%for a desired parking location, we will be able to mine historical data for
%how long someone in that situation will have to wait for a parking spot and
%make recommendations accordingly.
%
%Integration with Google Maps or other navigation solution will permit
%automatic retrieval of labeled search preferences in lieu of manual user
%input.
%
%
%Over time, accuracy and popularity of PocketParker should mutually enhance
%each other.  Increasingly dense data, stemming from higher user adoption rates
%and additional historical information, 
%
%...will enhance accuracy, and hence credibility of the application, with end users...
%
%
%labelled searches...
