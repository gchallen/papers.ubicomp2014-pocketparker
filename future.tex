\section{Future Work}
\label{sec-future}

\XXXnote{GWA: Street parking!}

\XXXnote{GWA: Better rate estimation using historical data.}

PocketParker presently bases its parking preditions on a fifteen-minute 
limited rolling window of recent parking events. We do not tap the benefit
of daily and weekly patterns that would otherwise enhance predictive accuracy.

In particular, currently available data does not address the situation of a
lot that fills abruptly, a typical situation on academic campuses during
class changes. Basing negative a negative recommendation about a particular
lot's availability solely upon recently acquired unsuccessful searches

A time lag necessarily exists between when users start
discovering on their own that a lot is full and when we have collected enough
data to conclude that a lot is an unwise recommendation.

, a number of users may already have been 

Knowing when and where users park on an ongoing basis would allow us to tune
our determination of whether a particular lot is approaching capacity well
in advance.  


...will enhance accuracy, and hence credibility of the application, with end users...

accuracy of our predictions of particular lots.

Any future enhancement of the app to solicit user input regarding desired
destinations would also significantly benefit from historical data---we
could suggest lots to end users 

When combined with user

Longer term historical data
would enhance predicion



\XXXnote{GWA: Integration into navigation software.}

Both the accuracy and usage of ParkerPocket would be enhanced significantly
by integration with existing mobile navigation solutions. \XXX{Google Maps by
name?} Access to large blocks of data of many users over extended time
would clearly address the present issue of lack of historical data.

[tie in with above segment...]

labelled searches...

\XXXnote{GWA: Incorporating user preferences.}

\XXXnote{GWA: Discuss what happens if all lots are full.}

\XXXnote{GWA: Hierarchical decomposition of lots over time.}




