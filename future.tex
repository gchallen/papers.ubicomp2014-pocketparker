\section{Limitations and Future Work}
\label{sec-future}

The pocketsourcing approach taken by PocketParker makes it easy to integrate
into existing mapping applications, such as Google Maps. Doing so would
benefit PocketParker in two ways. First, Google Maps and other navigation
tools are in extremely wide deployment, with the Play Store estimating
100~500~million installs for Google Maps. If we can increase the monitored
fraction significantly, much of the work PocketParker does to perform
estimation and work around low monitored fractions will be unnecessary.

PocketParker will also benefit from the increased amount of location context
available through integration into mapping software. We imagine a ``Help Me
Park'' button which engages PocketParker. This small piece of natural user
input allows PocketParker to identify \textit{explicit} searches and use them
to build up a lot desirability model with requiring annotations. However,
once PocketParker begins guiding users to available parking spaces we will
have to incorporate the effects of our guidance on natural user behavior.
However, we believe that many users will only query PocketParker when parking
in unfamiliar locations, while still providing data about lots they use
regularly and know well.

PocketParker presently bases its parking preditions on a fifteen-minute
limited rolling window of recent parking events. We do not presently tap the
benefit of daily and weekly patterns that would otherwise enhance predictive
accuracy, but hope to do so in the future. Maintaining a database of
previously collected historical data from our own application will increase
the sample size and hence statistical accuracy of our parking preditions.
This is another area where integration with a mapping application would help,
providing PocketParker with access to much more data.

Access to historical data will also address a present fundamental limitation:
the situation of a lot that fills abruptly, a typical occurence at
universities during class changes. Basing a negative recommendation about a
particular lot's availability solely upon recently acquired unsuccessful
searches implies that a time lag necessarily exists between when users start
discovering on their own that a lot is full and when we have collected enough
data to conclude---somewhat belatedly---that a lot is an unwise option.
Having historical data on hand will dissolve this limitation immediately:
using previous trends, we will be able to time parking advisories for
particular lots before they hit capacity.

Finally, we believe that once users begin interacting with PocketParker we
will see different preferences emerge. Some user will want PocketParker to
help them aggressively hunt for spots, and be willing to wait for drivers to
leave. Others may be more interested in simply finding a spot quickly even if
it is farther away. PocketParker has several parameters that can control its
predictions, and we will need to determine which are the most intuitive to
users.

%
%An anticipated future enhancement of our appliction, adding capacity to
%solicit the exact prefered parking destination from a user, will also
%significantly benefit from historical data. We will be able to suggest lots
%available to end users based on proximity to logistically preferable but full
%lots. If a user further furnishes insight into how long he is willing to wait
%for a desired parking location, we will be able to mine historical data for
%how long someone in that situation will have to wait for a parking spot and
%make recommendations accordingly.
%
%Integration with Google Maps or other navigation solution will permit
%automatic retrieval of labeled search preferences in lieu of manual user
%input.
%
%
%Over time, accuracy and popularity of PocketParker should mutually enhance
%each other.  Increasingly dense data, stemming from higher user adoption rates
%and additional historical information, 
%
%...will enhance accuracy, and hence credibility of the application, with end users...
%
%
%labelled searches...
