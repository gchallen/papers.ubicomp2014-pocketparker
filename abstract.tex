\subsection*{Abstract}

Searching for parking spots wastes time and generates frustration and harmful
vehicle emissions. To address these parking problems, we present
\textit{PocketParker}, a crowdsourcing system that uses smartphones to
predict parking lot availability. We consider PocketParker to be an example of
a subset of crowdsourcing we call \textit{pocketsourcing}. Pocketsourcing
applications require no explicit user input or additional infrastructure, and
can run effectively without the phone leaving the user's pocket. Users
interact with PocketParker only when looking for parking spots.

PocketParker detects parking and leaving events by monitoring the
smartphone's accelerometer and using a simple and energy-efficient activity
detection algorithm. Detected events are sent to the PocketParker server,
which incorporates them into its parking lot availability model, allowing it
to make predictions in response to availability queries. By estimating the
number of \textit{hidden drivers}---those not using PocketParker---we can use
monitored drivers to estimate arrival and departure rates and make accurate
predictions.

Our evaluation uses multiple data sets to determine the accuracy of each
PocketParker component and the system as a whole. We show that PocketParker
can accurately and rapidly detect parking events, and that our availability
estimator is accurate and robust to the presence of hidden drivers. Finally,
we use camera monitoring of several parking lots and a small-scale deployment
to demonstrate PocketParker's performance in the wild.
