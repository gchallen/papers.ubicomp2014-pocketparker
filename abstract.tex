\begin{abstract}
% <wc:start description="Abstract" max=150>
Searching for parking spots generates frustration and pollution. To address
these parking problems, we present \textit{PocketParker}, a crowdsourcing
system using smartphones to predict parking lot availability. PocketParker is
an example of a subset of crowdsourcing we call \textit{pocketsourcing}.
Pocketsourcing applications require no explicit user input or additional
infrastructure, running effectively without the phone leaving the user's
pocket. PocketParker detects arrivals and departures by leveraging existing
activity recognition algorithms. Detected events are used to maintain per-lot
availability models and respond to queries. By estimating the number of
drivers not using PocketParker, a small fraction of drivers can generate
accurate predictions. Our evaluation shows that PocketParker quickly and
correctly detects parking events and is robust to the presence of hidden
drivers. Camera monitoring of several parking lots as 105~PocketParker users
generated \num{10827}~events over 45~days shows that PocketParker was able to
correctly predict lot availability 94.2\% of the time.
% <wc:end>

\end{abstract}

\keywords{
  Smartphone sensing; Crowdsourcing; Parking.
}

\category{C.2.4}{Computer-Communication Networks}{Distributed Systems}
