\clearpage
\newpage

\section{Parking Lot Availability Estimation}
\label{sec-model}

In order for parking events to be useful, they must be incorporated into a
model allowing us to predict parking lot availability. Our goal is to respond
to queries with the probability that a given parking has a space available,
information that can be used in several ways to determine what lots to search
and in what order. In this section, we present the design of the PocketParker
parking lot availability estimator. PocketParker's estimator uses the events
produced by our parking event detector both to estimate the rates at which
drivers are searching and departing from the lot and to adjust the
availability probability directly.

\subsection{Overview}

Figure~\ref{fig-lots} shows an example setup with two parking lots and two
points of interest used throughout this section. For each lot $l$,
PocketParker maintains a time-varying probability that the lot has $n$ free
spots $P_l(t, X = n)$. While we are mainly interested in the probability that
the lot has a space available $P_l(t, X > 0)$, we maintain separate
probabilities for each number of free spots so that we can manipulate
individual probabilities in response to events and queries as described
below. We bound the count probability distribution to lie between 0 and the
capacity of the parking lot. Section~\ref{subsec-capacity} briefly describes
how PocketParker estimates lot capacity.

PocketParker's estimator receives two types of events: arrivals and
departures. However, for each arrival in a given lot, a number of additional
lots may have been searched unsuccessfully, information critical to the
accuracy of our availability model. Section~\ref{subsec-synthetic}
describeshow PocketParker uses the arrivals and a desirability model for each
lot to estimate implicit search behavior each arrival.
Section~\ref{subsec-online} further describes how we incorporate each kind of
information into our availability model.

Between events we want to maintain our availability model by estimating the
rate at which departures and searches are taking place. PocketParker must use
the events it can detect to estimate the rate at which events are taking
place in the lot, which includes the effect of drivers not using
PocketParker, which we call \textit{hidden drivers}. Accomplishing this
requires that we estimate the ratio between monitored and hidden drivers, and
we describe an approach to doing so in Section~\ref{subsec-hidden}. With an
estimate of the hidden driver ratio, we can scale the search and departure
rates according, described in Section~\ref{subsec-rates}.

\subsection{Estimating Lot Capacity}
\label{subsec-capacity}

PocketParker requires an estimate of lot capacity $C$ in several places.
First, we use this estimate to bound $P_l(t)$ such that $P_l(t, X > C) =
0\;\forall\;t$. Second, we use the capacity to determine the number of hidden
drivers, explained in more detail in Section~\ref{subsec-hidden}.

Recall from Section~\ref{FIXME} that our false-positive filter uses knowledge
of the location of lots obtained from the OpenStreetMap database. We estimate
the capacity of each lot by converting the location of the lot into a size
and dividing by the size of an average parking spot. \XXXnote{Anand and
Taeyeon, add capacity estimation stuff here. What is the size of the spot
that we determined? Reference for that. Estimates for each of the lots we
used and comparison with the true counts.} Errors in the capacity can result
if the size of parking spots in the lot differ from our estimate, or if the
parking lot is not efficiently packet with spots. Given the incentive of
parking lot designers to maximize capacity, we believe that the second case
will be unlikely. Parking spot sizes, however, may vary significantly from
lot to lot or based on the lot's location. To improve our estimate, we may
need to incorporate location-specific parking spot size estimates.
Alternatively, mapping databases may be directly annotated with the number of
spots per lot.

\subsection{Implicit Searches}
\label{subsec-synthetic}

PocketParker's detector identifies only arrivals and departures. However,
understanding and incorporating search behavior is critical to our model. For
example, if we observe the arrival rate fall at a given lot, it may be
because the lot is full, or it may be simply because fewer drivers are
arriving and the lot still has many spaces available.

In order to estimate search behavior, we need to understand the relationships
between parking lots, a small amount of additional information. 

\subsection{Online Updates}
\label{subsec-online}

Each arrival and departure received represent strong positive information:
moments when PocketParker knows either that a spot just existed (arrival) or
now exists (departure). Unsuccessful searches, in contrast, represent weaker
negative information, either because they may not have actually been observed
by PocketParker (unannotated) or so may not have actually taken place, or
because they may not have been thorough (annotated).

\subsection{Hidden Driver Estimation}
\label{subsec-hidden}

Maintaining an accurate count requires estimating the
percentage of drivers using the lot that are monitored by PocketParker, which
we discuss next.

\subsection{Rate Estimation}
\label{subsec-rates}
