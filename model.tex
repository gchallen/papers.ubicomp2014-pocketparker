\begin{figure}
\centering
\includegraphics[width=0.6\columnwidth]{./figures/CartoonLot.pdf}

\caption{\textbf{Example parking lot setup.} Two lots and three
destinations are shown.}

\label{fig-lots}
\end{figure}

\section{Availability Estimation}

In order for parking events to be useful, they must be incorporated into a
model that allows us to predict where parking is available. Because
PocketParker focuses on monitoring surface lots, not on-street parking, we
structure our prediction engine to return the probability that a given
parking lot has space available. This information is used by drivers to
determine what lots to search and in what order. PocketParker's estimator
uses the events produced by our parking event detector both to estimate the
rates at which drivers are searching and departing from the lot and to adjust
the availability probability directly. In this section, we present both the
design of the PocketParker client parking lot availability estimator and
portions of the backend server for our system.

\subsection{Overview}

Figure~\ref{fig-lots} shows an example setup with two parking lots and two
destinations that are used throughout this section. For each lot PocketParker
maintains a time-varying probability that the lot has $n$ free spots $P(t,
n)$. While we are mainly interested in the probability that the lot has a
space available $P_{free} = \sum_{n > 0} P(t, n)$, we maintain separate
probabilities for each number of free spots so that we can manipulate
individual probabilities in response to events and queries as described
below. We bound the count probability distribution to lie between 0 and the
capacity of the parking lot. 

PocketParker's estimator receives two types of events: arrivals and
departures. However, for each arrival in a given lot, a number of additional
lots may have been searched unsuccessfully, information critical to the
accuracy of our availability model. In the next two sections we describe how
PocketParker determines relationships between parking lots and combines that
information with arrivals to estimate implicit search behavior.

Between events we want to maintain our availability model by estimating the
rate at which departures and searches are taking place. PocketParker must use
the events it can detect to estimate the rate at which events are taking
place in the lot, which includes the effect of drivers not using
PocketParker, which we call \textit{hidden drivers}. Accomplishing this
requires that we estimate the ratio between monitored and hidden drivers.
With an estimate of the hidden driver ratio, we can scale the search and
departure rates accordingly. Finally, we integrate all of this information to
update our availability estimate as arrival and departure events are
received.

\subsection{Estimating Lot Capacity}

PocketParker requires an estimate of lot capacity $C$ in several places.
First, we use this estimate to bound $P(t)$ such that $P(t, n > C) =
0\;\forall\;t$. Second, we use the capacity to determine the number of hidden
drivers. To calculate a lot capacity, we use the location of the parking lot
obtained from the OpenStreetMap database~\cite{openstreetmap}. We derive the
lot size from its location and then divide the total size by that of a
typical standard parking spot lot design~\cite{parkingdesign}. For the three
lots monitored by our deployment, capacity estimates were all within 6\% of
manually-counted ground truth. Errors in the capacity can result if the size
of parking spots in the lot differ from our estimate, or if the parking lot
is not efficiently packed with spots. Given the incentive of lot
designers to maximize capacity, we consider the second case unlikely.

\subsection{Lot Relationships}

While PocketParker's parking event detector identifies only arrivals and
departures, identifying unsuccessful searches is crucial in order to
determine the reason for a drop in arrival rates. If we observe the arrival
rate fall at a given lot, it may be because the lot is full, or it may be
simply because fewer drivers are arriving and the lot still has many spaces
available. Observing unsuccessful searches in the first case allows
PocketParker to infer that the lot is full and suggest drivers park
elsewhere. In order to estimate search behavior, we need to understand the
relationships between parking lots. This requires two additional pieces of
data about each lot: what destinations it serves and how desirable it is.

\subsubsection{Lot destinations}

The lot destination represents the place or places where the user is
ultimately going after parking. In Figure~\ref{fig-lots}, lot~1 may be
associated with destinations A, B and C; while lot~2 is only linked to B.
While mapping software can be used to assign lots to the nearest labeled
building, this approach fails when lots serve multiple destinations. To
handle this case, PocketParker uses Wifi localization of the first access
point seen by the smartphone after the user parks to determine what indoor
location the user entered after parking. The probability distribution that
emerges from a history of these events can be used to predict where a user is
going at the moment that a parking event is detected. In the future, data
from navigation tools may be able to link destinations automatically with
lots by noting where users park after requesting directions to a particular
location.

\subsubsection{Desirability index}

The desirability index reflects a lot's relative preference to drivers. We
infer a lot's desirability from the destinations associated with each lot and
the lot's distance to each, assuming that PocketParker users prefer the
closest available lot to their final destination. In Figure~\ref{fig-lots},
if lot~2 is associated with destination~A it will be ranked less desirable
than lot~1 because it is further from the destination. Integration with
navigation tools can also help refine the desirability index by observing
what lots are searched by users on their way to a particular destination.
Currently PocketParker saves energy by enabling GPS only after detecting
parking events and so does not have a trace of the users locations before
parking that could be used to identify more desirable lots.

\subsection{Implicit Searches}

With an understanding of lot relationships we can use observed arrivals to
model implicit---or unobserved---searches. When a user parks in a given lot,
we use the desirability index of the lot to add unsuccessful searches in more
desirable lots associated with the same destination. There are two challenges
to this approach. First, as described above, lots may be associated with
multiple destinations. Second, the user may not have actually performed the
search. After discussing both of these issues below, we continue by
describing how PocketParker incorporates the information from implicit
searches in a way sensitive to these uncertainties.

\subsubsection{Determining the destination}

If a lot is associated with multiple destinations, we cannot immediately
determine the user's destination.  This is not a problem as long as all
potential destinations are on the same side of the lots.  For example, in
Figure~\ref{fig-lots}, if lots 1~and~2 are both associated with destinations
A~and~C, but not with B, then an arrival with an unknown destination into lot~2
can always be used to generate an implicit search in lot~1, since the
destination does not alter the desirability ranking for the two lots.

However, having two or more destinations that are located on different sides of
lots produces an ambiguity.  If both lots~1~and~2 are associated with all
destinations, then an arrival in lot~2 cannot be resolved directly. If the
user's destination was A, it may mean that lot~1 was searched and is full.
If the destination was B, the parking event may not indicate anything about
lot~1. To resolve this ambiguity, PocketParker uses information about the users
final destination gathered as described above.

\subsubsection{Speculative searches}

If we do not directly observe a user searching a lot before we detect an
arrival, we cannot be certain that they performed the search. If the unsearched
but preferable lot was available, they may not have searched it because they
preferred to choose the first available spot, enjoyed the exercise of walking
farther to their destination. However, these are not the type of users we
believe would benefit from or use PocketParker, since finding a
non-optimal parking spot is fairly simple in most cases.

A more interesting case is where a user has not performed a search in a
desirable lot because it \textit{looks} full. Users that park regularly at
the same destination may maintain temporal models for the availability of
spots in certain lots (``I can never park there after 9AM'') causing them to
discard those lots without searching them if they believe the probability of
finding a spot in the desirable lot is low. While this behavior can cause
users to miss available spots, these speculative searches are useful inputs
since they reflect lots users think are full.

A final corner case that PocketParker does not handle is if all lots for a
destination are full and many undetected unsuccessful searches are taking
place. On one hand, if all lots are full then spot availability is entirely
determined by departures and so search data is useless. On the other hand, we
would like to identify this situation for users that would prefer to avoid
destinations where it is impossible to park. Later we point out how
integrating PocketParker into existing navigation applications could address
this problem by making searches explicit.

\begin{figure}
\centering
\includegraphics[width=\columnwidth]{./simulator/figures/capacity.pdf}

\caption{\textbf{Example of capacity estimation.} Running counts for two lots
are shown.}

\vspace*{-0.2in}
\label{fig-capacityexample}
\end{figure}

\subsection{Hidden Driver Estimation}

Monitored PocketParker users compete for parking spaces with unmonitored
users, which we call \textit{hidden drivers}. While we assume that
PocketParker users are generally representative of the entire driving
population, we do not assume that all or even a large fraction of drivers
will download and install PocketParker. We want our system still to provide
accurate predictions with the limited information caused by hidden drivers.
To accomplish this, PocketParker needs to estimate the percentage of drivers
that are monitored, which we call the \textit{monitored fraction} $f_m$. A
low monitored fraction indicates that few users are using PocketParker,
and vice versa. Put another way, the amount
of uncertainty PocketParker faces when predicting availability is
inversely-proportional to the monitored fraction.

\subsubsection{Importance of monitored fraction estimation}

Two examples will illustrate why we need this information and how it is used.
First, when a monitored driver leaves a parking lot, the monitored fraction
determines how long PocketParker will predict that a spot in that lot is
available. As the monitored fraction increases, the probability of
PocketParker seeing the arrival into the lot that occupies that spot
increases, and we can increase the amount of time that we estimate a spot is
available. On the other hand, as the monitored fraction decreases we see
fewer arrivals and are faced with more uncertainty. Hence, PocketParker
reduces the amount of time it predicts the spot is available. Second,
PocketParker uses the arrival and departure rates of monitored drivers to
estimate changes to parking lot availability over time. Here we must scale
the observed number of events to the actual number of events, which requires
an estimate of the monitored fraction.

PocketParker estimates the monitored fraction by first determining the
monitored capacity---the capacity of the lot measured by monitored
drivers---and then using our estimate of the lot capacity. Specifically,
given a lot with capacity $C$, the monitored fraction can be estimated as
$f_m = \frac{C_m}{C}$. Our task then becomes estimating the monitored
capacity $C_m$. To estimate the monitored capacity we maintain a running
count $a$ for each lot, decremented when drivers arrive and incremented when
they leave. We can consider $a$ as a estimate of the number of spots
available in the lot scaled by $f_m$, although we do not bound $a$ as $0 \le
a \le C$.

Figure~\ref{fig-capacityexample} shows an example of the running count for
two related lots over seven days using data generated by our lot simulator
described in more detail in the evaluation. Both lots have capacity 200 and
the actual monitored fraction is 0.1. As the data shows, the running count
experiences long-period (greater than one day) fluctuations due to events
missed by our event detector and the randomness associated with the small
percentage of drivers being monitored. However, the data also contains
short-period (less than one day) fluctuations caused by the dynamics of the
lot being monitored, and these fluctuations are roughly the size of the
monitored capacity $C_m$, which in this case is 20 spots.

This observation motivates the design of our monitored capacity estimator.
First, we bin the data into 24~hour intervals. Next, we identify the largest
availability swing over each window. Finally, we average multiple swings
together for a period of days to determine the final estimate. This simple
approach works well on lots that fill on a regular basis. For the example in
Figure~\ref{fig-capacityexample}, our estimator estimates the monitored
capacity of lots~1~and~2 as 21.01 and 21.08, respectively, within 10\% of the
true value in both cases. We perform a further analysis of our capacity
estimator using multiple lot simulations in the evaluation.

For lots that do not fill regularly, we may need to produce a
weighted sum where larger swings are weighted more heavily given our
assumption that they more accurately measure the true monitored lot capacity. 
Another approach is to use the $f_m$ estimated at
desirable lots for a given destination, which are more likely to fill
completely and often, to estimate the $f_m$ for
lesser desirable lots. Here we are making the reasonable assumption that lots
connected to the same destination share similar fractions of PocketParker
users. Finally, PocketParker's monitored fraction estimator runs periodically
to incorporate changes in the monitored fraction caused by increasing use of
PocketParker.

\subsection{Rate Estimation}

When PocketParker receives arrival and departure event information, it knows
something concrete about the state of the lot. However, to predict
availability at other times we need to adjust our estimation based on
recently-observed events, which we call rate estimation. To estimate the rate
of events in the entire population including hidden drivers, PocketParker
must scale its rate of parking events by monitored drivers appropriately.
Next, we use these scaled estimates to adjust the probability that a lot has
a certain number of spots and spots available.

During a time interval $t_0$ to $t_1$, PocketParker will observe some number
of searches $s_{obs}(t_0, t_1)$ or departures $d_{obs}(t_0, t_1)$ in any
given lot\footnote{Without loss of generality our examples of scaling and
estimating rates use notation for the search rate.}. Note that the search
count includes both arrivals---successful searches---and implicit
unsuccessful searches derived from arrivals at related lots as explained
above. However, depending on the monitored fraction $f_m$ the true count
$s_{true}(t_0, t_1)$ is likely to be much larger. Rather than simply scaling
the count by $\frac{1}{f_m}$, we want to determine the probability
distribution over all possible true counts given the rate we observed and the
estimated monitored fraction. One reason we do not simply scale by
$\frac{1}{f_m}$ is that our uncertainty about the true count should be
affected by $f_m$. If all drivers use PocketParker, we know the true count
exactly; if few do, we should be uncertain.

To compute the probability distribution we treat $s_{obs}$ as the output of a
binomial distribution with probability $f_m$ and vary the number of trials.
The binomial distribution reflects the fact that drivers are either monitored
by PocketParker or not with estimated probability $f_m$. Specifically:
%
\[
%
P(s_{true}| s_{obs}) = C \cdot {s_{obs} \choose s_{true}}
f_m^{(s_{obs})} \cdot (1 - f_m)^{(s_{true} - s_{obs})}
%
\]
%
where $C$ is a renormalization constant equal to $\sum_{s_{true}} P$.

\subsubsection{Updating the count probabilities}

Given the probability that a lot has $n$ free spots at time $t_0$, $P(t_0,
n)$, we want to estimate the probabilities $P(t_1, n)$ at a later time $t_1$.
PocketParker uses recently-observed arrivals, implicit searches and
departures to estimate the search $s_{est}$ and departure $d_{est}$ rates the
lot experienced between $t_0$ and $t_1$. Currently, we use arrival and
departures over a fixed-size window $I$ before $t_0$, $s_{obs}(t_0 -
I, t_0)$ scaled to the length of $t_0$ to $t_1$:
%
\[s_{est}(t_0, t_1) = s_{obs}(t_0 - I, t_0) \cdot \frac{(t_1 - t_0)}{I} \]
%
The value of $s_{est}(t_0, t_1)$ is then scaled as described above to
determine the distribution of $s_{true}$.  Given the predictable traffic flows
of our campus environment over the course of a term, PocketParker assumes the
rates experienced over the last $I$ time interval will continue. It may be
possible to perform better rate estimation by using historical information,
but this is left as future work.

The distribution of search rates $s_{true}(t_0, t_1)$ represents the
probabilities that the number of available spots in the lot will decline,
whereas the departure rate $d_{true}(t_0, t_1)$ represents the probability
the number of spots will increase due to departures. The convolution of $-1
\cdot s_{true}$ and $d_{true}$, $\Delta(t_0, t_1)$, represents the change in
the number of spots produced by the specific combination of arrival and
departure rates. A further convolution of $\Delta(t_0, t_1)$ with $P(t_0,
n)$ produces $P(t_1, n)$, the probability at $t_1$:
%
\[ P(t_1, n) = P(t_0, n) * (-1 \cdot s_{true}(t_0, t_1) * d_{true}(t_0,
t_1)) \]
%
where $*$ represents the discrete convolution.

Note that the convolution of $P$ with $\Delta$ can cause non-zero
probabilities in $P$ that violate our boundary conditions, namely that
$P(n < 0) = 0$ and $P(n > C) = 0$ where $C$ is the estimated capacity of
the lot. To correct this, we simply set $P(n = 0) = \sum_{n < 0} P(n)$
and $P(n = C) = \sum_{n > C} P(n)$, assigning all the probability that
the lot has less that zero free spots to the zero state and all probability
that it has more than the capacity of the lot of free spots to the empty
state.

\subsubsection{Rateless spreading}

If the departure rate exceeds the arrival rate, the probability mass of
$\Delta$ will lie primarily to the positive side and it will shift $P$ in the
positive direction, producing higher probabilities that spots are available
in the lot and lowering the probability that the lot is full. The opposite is
true when the search rate exceeds the arrival rate.

An important case is intervals during which PocketParker has observed neither
arrivals nor departures in a given lot. In this case, $\Delta$ will be
centered around $0$ but have a spread determined by the monitored fraction.
Its effect on $P$ will be to redistribute the probability mass more evenly
across the entire interval from $0$ to $C$. Taken over many intervals, the
probability of the lot having any number of spots available will equalize,
which is what we would expect: after a long period without any information,
all states become equally likely and we cannot make an accurate prediction of
the state of the lot. Note also that the speed at which the probabilities are
redistributed through rateless spreading is determined again by the monitored
fraction. The fewer drivers we monitor, the more quickly we lose all memory
of the state of the lot.

\subsection{Online Updates}

Finally, we conclude by describing how PocketParker uses arrival to adjust
its availability model instantaneously at runtime. Each arrival and departure
received at time $t$ represent strong positive information---moments when
PocketParker knows either that a spot just existed (arrival) or now exists
(departure). PocketParker uses these events to adjust the probability
distribution and incorporate this new information. 

Arrivals provide two somewhat conflicting pieces of information. First,
PocketParker knows that at the time of the arrival there was a spot free, so
in this way arrivals indicate that the lot is not full. However, PocketParker
also knows that immediately after an arrival the lot has one fewer available
spots. So we incorporate arrivals in two steps. First, we set $P(t, 0) = 0$
indicating the availability of a spot and renormalize the distribution.
Second, we shift the entire distribution downward by one spot, $P(t, n) =
P(t, n - 1)$, reflecting the loss of a parking space due to the arrival.

Departures produce a straightforward change to the probability distribution.
When a user departs, we know at that moment that there is a free spot in the
lot, so we can set $P(t, 0) = 0$ and renormalize the distribution. Note that,
since the probability that the lot is free is $P_{free} = \sum_{n > 0} P(t,
n)$, at the exact time of each departure the probability that a spot is free
is equal to 1. 

Unsuccessful implicit searches, in contrast, represent weaker negative
information, both because they were not observed by PocketParker and so may
not have actually taken place, or because they may not have been thorough.
What we want is to increase the probability that the lot is full while
reflecting our current estimate of the lot. We do this by shifting the
availability distribution towards full by some amount $s$, which we refer to
as the \textit{search shift parameter}. So, after an implicit unsuccessful
search, we set $P(t, n) = P(t, n - s)$, with $P(t, 0) = \sum_0^s P(t, n)$.
The search shift parameter determines how aggressively PocketParker will use
information provided by implicit searches.

\subsubsection{Weighted arrivals and departures}

Shifting the distribution one space on arrivals and departures is the most
conservative approach representing what we definitely know: that one spot is
available. However, if we assume that our monitored drivers are
representative of some larger number of hidden drivers, we may set $P_l(t, n
< X) = 0$ for some $X$ larger than 1 and scaling with $\frac{1}{f_m}$. For
our experiments we choose the conservative approach and set $X = 1$. As
future work we consider how users may customize the behavior of PocketParker
to be more or less aggressive in locating parking spots, trading off time for
a better spot.
