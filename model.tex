\clearpage
\newpage

\section{Availability Estimation}
\label{sec-model}

In order for parking events to be useful, they must be incorporated into a
model allowing us to predict parking lot availability. Our goal is to respond
to queries with the probability that a given parking has a space available,
information that can be used in several ways to determine what lots to search
and in what order. In this section, we present the design of the PocketParker
parking lot availability estimator. PocketParker's estimator uses the events
produced by our parking event detector both to estimate the rates at which
drivers are searching and departing from the lot and to adjust the
availability probability directly.

\subsection{Overview}

\begin{figure}
\centering
\includegraphics[width=\columnwidth]{./figures/CartoonLot.pdf}

\caption{Example parking lot setup with two lots 1~and~2 and three
destinations A, B and C.}

\label{fig-lots}
\end{figure}

Figure~\ref{fig-lots} shows an example setup with two parking lots and two
destinations used throughout this section. For each lot $l$, PocketParker
maintains a time-varying probability that the lot has $n$ free spots $P_l(t,
X = n)$. While we are mainly interested in the probability that the lot has a
space available $P_l(t, X > 0)$, we maintain separate probabilities for each
number of free spots so that we can manipulate individual probabilities in
response to events and queries as described below. We bound the count
probability distribution to lie between 0 and the capacity of the parking
lot. Section~\ref{subsec-capacity} briefly describes how PocketParker
estimates lot capacity.

PocketParker's estimator receives two types of events: arrivals and
departures. However, for each arrival in a given lot, a number of additional
lots may have been searched unsuccessfully, information critical to the
accuracy of our availability model. Section~\ref{subsec-lots} describes how
PocketParker determines relationships between parking lots, and
Section~\ref{subsec-implicit} describes how we combine that information with
arrivals to estimate implicit search behavior. Section~\ref{subsec-online}
further describes how we incorporate each kind of information into our
availability model.

Between events we want to maintain our availability model by estimating the
rate at which departures and searches are taking place. PocketParker must use
the events it can detect to estimate the rate at which events are taking
place in the lot, which includes the effect of drivers not using
PocketParker, which we call \textit{hidden drivers}. Accomplishing this
requires that we estimate the ratio between monitored and hidden drivers, and
we describe an approach to doing so in Section~\ref{subsec-hidden}. With an
estimate of the hidden driver ratio, we can scale the search and departure
rates according, described in Section~\ref{subsec-rates}.

\subsection{Estimating Lot Capacity}
\label{subsec-capacity}

PocketParker requires an estimate of lot capacity $C$ in several places.
First, we use this estimate to bound $P_l(t)$ such that $P_l(t, X > C) =
0\;\forall\;t$. Second, we use the capacity to determine the number of hidden
drivers, explained in more detail in Section~\ref{subsec-hidden}.

Recall from Section~\ref{FIXME} that our false-positive filter uses knowledge
of the location of lots obtained from the OpenStreetMap database. We estimate
the capacity of each lot by converting the location of the lot into a size
and dividing by the size of an average parking spot. \XXXnote{Anand and
Taeyeon, add capacity estimation stuff here. What is the size of the spot
that we determined? Reference for that. Estimates for each of the lots we
used and comparison with the true counts.} Errors in the capacity can result
if the size of parking spots in the lot differ from our estimate, or if the
parking lot is not efficiently packet with spots. Given the incentive of
parking lot designers to maximize capacity, we believe that the second case
will be unlikely. Parking spot sizes, however, may vary significantly from
lot to lot or based on the lot's location. To improve our estimate, we may
need to incorporate location-specific parking spot size estimates.
Alternatively, mapping databases may be directly annotated with the number of
spots per lot.

\subsection{Lot Relationships}
\label{subsec-lots}

PocketParker's detector identifies only arrivals and departures. However,
understanding and incorporating search behavior is critical to our model. For
example, if we observe the arrival rate fall at a given lot, it may be
because the lot is full, or it may be simply because fewer drivers are
arriving and the lot still has many spaces available.

In order to estimate search behavior, we need to understand the relationships
between parking lots. This requires two additional pieces of data about each
lot: one or multiple destinations, and a desirability index. The destination
represents the place the user is going when they park in a given lot, and
note that some lots may be associated with multiple destinations. In
Figure~\ref{fig-lots}, lot~1 may be associated with destinations A, B and C;
while lot~2 is only linked to B.

The desirability index produces an ordering of lots associated with a given
point-of-interest based on how preferable they are compared with other lots.
We assume that most users will park in desirable lots if they are available,
and may have searched in more desirable lots before parking in a lot
desirable lot. In Figure~\ref{fig-lots}, if Lot~2 is associated with
destination~A it will probably receive a lower desirability score than Lot~1
because it is further away.

While this information is not currently part of open mapping databases, we
believe that it is straightforward to collect. Parking lot operators and
business owners can annotate the mapping database with destinations for each
lot. In addition, data from navigation tools may be able to automatically
link destinations with lots by noting where users park after requesting
directions to a particular place. The desirability index may also be
determined by navigation tools observing what lots are searched by users on
their way to a particular destination. Lacking these traces, simple proximity
to the destination may determine the desirability index directly. As example
of this automatic annotation, in Figure~\ref{lots} if both lot~1~and~2 are
associated with destination~A, we can consider lot~2 less desirable than
lot~1 because lot~1 lies between it and the destination.

\subsection{Implicit Searches}
\label{subsec-implicit}

With an understanding of lot relationships we can use observed arrivals to
model implicit---or unobserved---searches. When a user parks in a given lot,
we use the desirability index of the lot to add unsuccessful searches in more
desirable lots associated with the some destination. There are two challenges
to this approach. First, as described above, lots may be associated with
multiple destinations. Second, the user may not have actually performed the
search. After discussing both of these issues below,
Section~\ref{subsec-online} describe below how PocketParker incorporates the
information from implicit searches in a way sensitive to these uncertainties.

\subsubsection{Determining the destination}

If a lot is associated with multiple destinations, we cannot uniquely
determine the destination of the user. However, this only becomes important
if the two destinations would produce different desirability rankings for
affected lots. For example, in Figure~\ref{fig-lots}, if lots 1~and~2 are
both associated with destinations A~and~C, but not with B, then an arrival
with an unknown destination into lot~2 can always be used to generate an
implicit search in lot~1, since the desirability ranking for the two lots are
unchanged if the destination is either A~or~C. However, if both lots~1~and~2
are associated with all three destinations, then an arrival detected in lot~2
becomes more ambiguous. If the user was trying to go to destination~A, it may
mean that lot~1 was searched and is full; however, if they were trying to go
to destination~B, it may not indicate anything about lot~1. 

If lot destination annotations are generated by mapping software, we can use
this data to estimate the probability that a user is going to each of the
destinations associated with a particular lot. Instead of generating a single
implicit search in one lot, we generate multiple implicit searches in each of
the lots weighted by the destination probabilities. Lacking this arrival
data, we simply generate implicit searches in each destination associated
with a given lot.

\subsubsection{Speculative searches}

If we do not directly observe a user searching a lot before we detect an
arrival, we cannot be certain that they performed the search. If the
unsearched but preferable lot was available, they may not have searched it
because they prefer to choose the first available spot, enjoy the exercise of
walking farther to their destination, or want to irritate their passengers.
However, these are not the type of users we believe would benefit from or use
our PocketParker application, since finding a non-optimal parking spot is
fairly simple in most cases.

A more interesting case is where a user has not performed a search before
parking in a less-desirable lot because they \textit{believe} the more
desirable lot to be full. Users that park regularly at the same destination
usually have their own mental models for the availability of spots in certain
lots, causing them to discard those lots without searching them if they
believe the probability of finding a spot in the desirable lot is low. While
this behavior can cause users to miss available spots, these speculative
searches are useful inputs into our PocketParker model since it represents
what lots users think are full.

\subsection{Online Updates}
\label{subsec-online}

Each arrival and departure received represent strong positive information:
moments when PocketParker knows either that a spot just existed (arrival) or
now exists (departure). Unsuccessful searches, in contrast, represent weaker
negative information, either because they may not have actually been observed
by PocketParker (unannotated) or so may not have actually taken place, or
because they may not have been thorough (annotated).

\subsection{Hidden Driver Estimation}
\label{subsec-hidden}

Maintaining an accurate count requires estimating the
percentage of drivers using the lot that are monitored by PocketParker, which
we discuss next.

\subsection{Rate Estimation}
\label{subsec-rates}
